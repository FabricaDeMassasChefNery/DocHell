%%------------------------------------------------------------------------------
%%----------------Variaveis para o LaTeX no Relatório---------------------------
\newcommand{\universidade}{Universidade de Bras\'ilia - Campus Gama}
\newcommand{\hell}{Nome da Materia}		%nome do arquivo
% \renewcommand{\autor}{\ziul}							%criador do pdf - só observado nos detalhes do pdf
\newcommand{\assunto}{Experimento}			%Nome do experimento
\newcommand{\ver}{Titulo~- \assunto}						%numeração do experimento
\newcommand{\professor}{Professor}
\newcommand{\curso}{ENGENHARIA DE SOFTWARE}
\newcommand{\turma}{D}

%%------------------------------------------------------------------------------
%%------------------------Detalhes para serem revistos--------------------------
\newcommand{\keyw}{Palavras,  Chaves, \assunto}	%Keywords
\newcommand{\entrega}{\today}		%data de entrega do relatório


%%------------------------------------------------------------------------------
%%------------------------------Nomes-------------------------------------------
\newcommand{\luiz}{\href{http://lattes.cnpq.br/1109478949026592}{Luiz Fernando Gomes de Oliveira}}	%Para link
\newcommand{\ziul}{{Luiz Fernando Gomes de Oliveira}}					%Sem link
\newcommand{\luizmatricula}{10/46969}
\newcommand{\eluiz}{ziuloliveira@gmail.com}

%----------------------Foto e bibliografia do(s) autor(es)----------------------
%%------------------------------------------------------------------------------
%					  Deve de ser COLADO ao final do texto
%%------------------------------------------------------------------------------
%	\begin{IEEEbiography}[{\includegraphics[width=1in,height=1.25in,clip,keepaspectratio]{./fts/luiz}}]{\luiz}\label{luiz}
%	É, sou eu. Aparecendo aqui só de brinks. Meio que trollando um relatório.
%	\end{IEEEbiography}
%
%	\begin{IEEEbiography}[{\includegraphics[width=1in,height=1.25in,clip,keepaspectratio]{ffuu}}]{Helbert Junior}\label{panda}
%	Pow, eu tinha que aparecer também né?
%	\end{IEEEbiography}
%%------------------------------------------------------------------------------

%%------------------------------------------------------------------------------
%%-----------------------------Organização--------------------------------------
\makeatletter
\@ifclassloaded{scrartcl}% Slide
{
	\newcommand{\names}{\luiz}
	\newcommand{\namecapa}{\ziul \\ \ziul}
	%\newcommand{\allmatriculas}{\luizmatricula,\canelamatricula}
}
{
	\newcommand{\names}{\luiz}
	\newcommand{\allmatriculas}{\luizmatricula}

}
\makeatother


\newcommand{\cor}{vermelho}
\newcommand{\comprimeto}{20m}
\title{\ver}
\author{\names}
\date{\entrega}



%%------------------------------------------------------------------------------
% Cabeçalho das páginas, se tiver no modelo
\markboth{Universidade de Bras\'ilia - Campus Gama - FGA, \entrega}
{Shell \MakeLowercase{\textit{et al.}}: \ver}
