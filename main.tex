\documentclass[journal,compsoc]{IEEEtran}\newcommand{\journal}{true}
% \documentclass[conference]{IEEEtran}\newcommand{\journal}{false}
% \documentclass[12pt,openright,twoside,a4paper,english,french,spanish]{abntex2}
% \documentclass[a4paper,14pt]{report}
% \documentclass[a4paper]{coursepaper}
% \documentclass[12pt,answers]{exam}
% \documentclass[a4paper,11pt]{article}
% \documentclass[a4paper,11pt]{book}
% \documentclass[paper=256mm:192mm,fontsize=20pt,pagesize,parskip=half-]{scrartcl} %deprected
\input packages

\input nomes
\input setup

\begin{document}

\makeatletter
% \@ifclassloaded{IEEEtran}{}
\makeatother

\onecolumn    %Conferir se pode ser feito em duas colunas
\imprimeabstract %Dependendo do modelo, tem que ficar antes ou depois de \maketitle
\maketitle
% \imprimeabstract %Dependendo do modelo, tem que ficar antes ou depois de \maketitle



%\pagestyle{myheadings}
\setcounter{page}{1}
\pagenumbering{roman}

  \selectlanguage{brazil}   %voltando para o português.
  
  %\tableofcontents     % Sumario
  %\listoffigures
  %\listoftables
  %\lstlistoflistings

%\newpage


\setcounter{page}{1}
\pagenumbering{arabic}
% \thispagestyle{empty}
% \IEEEdisplaynotcompsoctitleabstractindextext
% \IEEEpeerreviewmaketitle
  
%\onecolumn   %Para texto em uma coluna
%\twocolumn   %Para texto em duas colunas
\begin{multicols}{2} % Insira dentro das {}  aquantidade de colunas desejadas
%\begin{alphasection}
  \input conteudo/conteudo
  \rowcolors{0}{white}{white}
%\end{alphasection}
\end{multicols}
%\onecolumn
%\printglossaries
%\glsdescwidth


%%% Para textos em paralelo, utiilize:

% \begin{Parallel}{7cm}{7cm}
%   \ParallelRText{Texto 1}
%   \ParallelLText{Texto 2}
% \end{Parallel}


\glossarystyle{altlist}
\providetranslation{Notation (glossaries)}{Notação}
\providetranslation{Description (glossaries)}{Descrição}
\phantomsection
\addcontentsline{toc}{chapter}{Gloss\'ario\label{glossary}}
% \setlength{\glsdescwidth}{0.48\textwidth}%
\printglossary[title=Gloss\'ario]
% \newpage
\phantomsection
\addcontentsline{toc}{chapter}{Lista de Termos}\label{acrom}
\printglossary[type=\acronymtype,title=Lista de Termos,toctitle=Lista de Termos]


\glsaddall


  \bibliographystyle{abnt-num}%{IEEEtran}%{abnt-num}%{ieeetr}%{abnt-alf}
  \bibliography{bibliography}   % expects file "bibliography.bib" 

\end{document}
