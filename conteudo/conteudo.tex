\newcommand\tab[1][1cm]{\hspace*{#1}}

\section{Introdução}\label{sec}
\tab O presente trabalho tem como finalidade apresentar o resultado da Engenharia de Requisitos da Fábrica de Massas do Chef Nery. O empreendimento trata-se de uma micro empresa que fabrica diferentes tipos de massas.\\
\tab O contexto relacionado a necessidade do projeto é principalmente relacionado a necessidade de um melhor gerenciamento de clientes visando então buscar uma melhor forma de gerenciar seus pedidos e divulgar seus produtos.\\
\tab Para isso, é de fundamental importancia utilizar uma metodologia para o levantamento de requisitos e gerenciamento do processo em questão.\\
\tab Com uma análise acerca do contexto foi utilizado uma abordagem ágil composto de atividades do Scaled Agile Framework (SAFe). Com isso, foi possível desenhar o Processo de Engenharia de Requisitos contendo um também um modelo do Processo que será implementado no Trabalho 2.


\section{Contexto da Empresa (Chef Nery)} % (fold)
\label{sec:nova_sess_o}

\section{Justificativa da Abordagem}
\label{sec:nova_sess_o}

\tab A partir da dećada de 1970, a mudança tecnológica  possibilitou um crescimento exorbitante dos recursos de hardware e permitiram que produtos com tecnologias mais avançadas fossem criadas. Entretanto, as tećnicas utilizadas no desenvolvimento  de software não acompanharam o crescimento dos recursos de hardware, este fenômeno ficou conhecido com a Crise de Software. Neste âmbito, Dijkstra destaca:\\


\tab Não obstante,  ao longo dos anos, notou-se o aumento de metodologias de produção de software com objetivo de aumentar a qualidade e eficiência neste processo. A IEEE define metodologias de desenvolvimento de software como uma abordagem sistemática obtida na etapa de desenvolvimento, no qual, o papel do engenheiro é assegurar que sejam realizadas as melhores escolhas naquele determinado contexto. Vale ressaltar, que estas escolhas têm implicações diretas no sucesso ou fracasso de um projeto.\\

\tab Na etapa de seleção de metodologia foi utilizada o fluxograma proposto pela empresa de consultoria Fortezza Consulting [N]. Apresentado abaixo: \\ 

\tab Desta forma, foram considerada as seguintes perguntas: \\

\begin{enumerate}
	\item \textsl{“Does the scope include software enablement of business processes?”} 
	\item \textsl{“Does the scope include software enablement of business processes?”}
	\item \textsl{“Is the scope lacking in specificity, and unlikely to remain stable?”} 
	\item \textsl{“Is the customer willing and able to offer flexibility on scope?”} 
\end{enumerate}

\tab Com base nas respostas das perguntas acima, foi identificada que a metodologia ágil se adaptaria melhor ao nosso contexto.  

\tab Além disso, uma série de fatores que justificaram o emprego da metodologia ágil, os quais estão descritos abaixo:

\begin{enumerate}
	\item \textsl{Flexibilidade a mudanças;} 
	\item \textsl{Comunicação com o cliente;}
	\item \textsl{Planejamento;} 
	\item \textsl{Equipe de desenvolvedores;}
	\item \textsl{Tamanho do Projeto;}
\end{enumerate}


{
	\large{Flexibilidade a mudanças\\}

	\tab O cliente, dono da fábrica de massas “Chef Nery”, ainda não alcançou uma idéia final concreta sobre o sistema que deseja adquirir. Os requisitos, embora já bem direcionados, permanecem instáveis no que diz respeito à definição de utilidades – ainda não é muito certo tudo o que será essencial ou apenas “bom ter” no projeto. Diante tal contexto, é natural que também o cliente esteja predisposto a flexibilizar o escopo da aplicação, permitindo que se assentem gradualmente as diretrizes definitivas do sistema. \\
	\tab Esta característica é favorável ao uso de metodologias ágeis, que preveem mudanças constantes no escopo e espera flexibilidade dos cliente envolvido [3]. \\
}

{
	\large{Comunicação com o cliente\\}

	\tab O manifesto ágil, criado por  em 1997, propõe uma série de princípios. Dentre eles, destaca-se a interação entre indivíduos como sendo mais relevante que os processos e ferramentas  [3]. Assim,  é necessário que os indivíduos tenham uma comunicação bastante próxima  com o cliente e que as interações com o mesmo possam propiciar mudanças e evolução constantes, possibilitando  a obtenção de requisitos voláteis. \\
	\tab O pequeno tamanho da equipe, que é formada por apenas 4 estudantes de Engenharia de Software e o cliente, somado ao fato de o cliente estar apto e disposto a participar de reuniões e atender as necessidades da equipe de desenvolvimento, possibilita a comunicação próxima da qual as metodologias ágeis se beneficiam. Para tanto, foram adotadas algumas ferramentas de comunicação como o WhatsApp e Hangouts (para situações de impossibilidade de encontros físicos ou necessidade de reuniões emergenciais). \\

}
	
{
	\large{Planejamento\\}

	\tab Em metodologias ágeis, usualmente, as equipes economizam um tempo significativo com o planejamento, visto que as interações que ocorrem junto ao cliente fornecem feedbacks constantes. A título de comparação, em metodologias tradicionais aproximadamente 20 por cento do tempo é gasto na etapas de planejamento e replanejamento [4]. \\
	\tab Não obstante, o cliente em nosso contexto não exige documentação expressiva, pelo contrário valoriza as interações provenientes da metodologia ágil. Desta forma, neste contexto  específico é mais interessante adoção da metodologia ágil,  pois esta  possibilita uma economia de recursos e tempo. \\

}

{
	\large{Equipe de desenvolvedores\\

	\tab A equipe de projetistas e desenvolvedores (que, na verdade, é a mesma equipe) já possui melhor familiaridade com metodologias ágeis, compreendendo relativamente bem suas etapas e rituais. \\
	\tab Somado a isto, há o tamanho diminuto da equipe, que consta 4 integrantes e o cliente, o que possibilita e também facilita uma comunicação intensa e frequente dos integrantes entre si e com o cliente. \\
	\tab Estas características tendem a uma menor dependência de formalidades em documentações e grande dinamicidade na gerência da equipe – ambos fatores que favorecem o uso de metodologias ágeis [5]. \\
}

{
	\large{Tamanho do Projeto\\}

	\tab De acordo com o manifesto ágil, é necessário que os indivíduos tenham uma comunicação bastante próxima com o cliente (BECK et al, 2001). Não obstante, em projetos menores a comunicação é privilegiada visto a maior interação entre os integrantes do projeto. \\
	\tab Além disso, deve-se destacar que em projetos mais complexos, usualmente,  é preferível a utilização de metodologias híbridas ou tradicionais, visto que estas detalham e documentam de forma mais precisa os processos e subprocessos. \\
	\tab Em projetos menores este formalismo é menos relevante, pois existe uma comunicação e um feedback maior entre os envolvidos. \\
}

{\large{Planejamento\\}}

\section{Processo de Engenharia de Requisitos}
\label{sec:nova_sess_o}

\section{Elicitação de Requisitos}
\label{sec:nova_sess_o}

\section{Tópicos de Gerenciamento de Requisitos}
\label{sec:nova_sess_o}

\section{Planejamento do Projeto}
\label{sec:nova_sess_o}

\section{Ferramentas de Gestão de Requisitos}
\label{sec}
\tab Foram realizadas pesquisas comparativas entre as ferramentas para gerir os requisitos, podendo elas serem versões Web ou ainda em versão de Desktop. Foram escolhidas então: Jira, Innoslate e Rally Dev.\\

{\large{8.1 Critérios}}\\

\tab Para esta análise, foram listadas 5 características julgadas importantes para a gestão de requisitos para o projeto, referentes à Usabilidade, Rastreabilidade, Gestão de Mudanças, Flexibilidade e Licença.\\
\tab \textbf{Usabilidade:} Analisa se a usabilidade da ferramenta é de fato intuitiva ou não, o que pode vir a trazer contratempos para a equipe.\\
\tab \textbf{Licença:} Analisa a licença da ferramenta, caso seja grátis, paga, se possui isenção para projetos educativos, etc.\\
\tab \textbf{Rastreabilidade:} Analisa se é possível rastrear de forma eficaz a ferramenta.\\
\tab \textbf{Gestão de Mudanças:} Analisa quais são as funcionalidades que ajudam na gestão de mudanças, análise de impacto, e escopo.\\
\tab \textbf{Flexibilidade:} Analisa a flexibilidade de personalização da ferramenta ao contexto do projeto.\\

{\large{8.2 Pontuação}}\\

\tab Para conseguir escolher com exatidão, foi atribuído uma pontuação a tais características com o intuito de esclarecer numericamente qual a ferramenta que nos auxiliaria. A pontuação foi ponderada de 0 à 5 sendo mais próximo de 5 muito pertinente às necessidades do projeto, e mais próximo de 0 muito divergente do fim das reais necessidades do projeto.\\

{\large{8.3 Resultados}}\\

\includegraphics[width=1\textwidth]{conteudo/resultados}\\

{\large{8.4 Descrição das notas atribuídas}}\\

\textbf{Usabilidade:}\\
	\tab Atlassian Jira: Interface bonita; Nomes bem intuitivos; Plugins gráficos interessantes.\\
	\tab Innoslate: Interface satisfatória; \\
	\tab RallyDev: Interface satisfatória; Ícones intuitivos;\\

\textbf{Licença:}\\
	\tab Atlassian Jira: Gratuito por 30 dias;\\
	\tab Innoslate: Versão grátis, com restrições;\\
	\tab RallyDev: Versão grátis, com restrições;\\

\textbf{Rastreabilidade:}\\
	\tab Atlassian Jira : Hierarquia pouco intuitiva; não existe ferramenta visual para controlar importância de requisitos em relação aos demais.\\
	\tab Innoslate: Rastreabilidade bastante intuitiva; Geração automática de índice de qualidade de requisitos e numeração na criação de entidade.\\
	\tab RallyDev: Hierarquia com fácil localização; Opção de linkar requisitos filhos;\\

\textbf{Gestão de mudanças:}\\
	\tab Atlassian Jira: Não aparenta ter algum feedback de mudanças do próprio autor ou de outros autores;\\
	\tab Innoslate: Controle de versão eficaz; Notificações com últimas alterações com autor e data;\\
	\tab RallyDev: Controle de versão eficaz;\\

\textbf{Flexibilidade:}
	\tab Atlassian Jira: Não apresenta ter opção de personalização;\\
	\tab Innoslate: Opções de fixar e desafixar abas importantes para o projeto;\\
	\tab RallyDev: Totalmente flexivel para modificações relevantes para o projeto com inúmeras possibilidades de plugins.\\

{\large{8.5 Escolha da ferramenta}}\\

\tab Após uma profunda análise em cada uma das ferramentas estudadas, foi decidido que o Rally Dev será utilizada para o gerenciamento de requisitos. Suas características de personalização e controle de versão foram fundamentais para a criação do próprio modelo de rastreabilidade do projeto.  \\


\section{Considerações Finais}
\label{sec:nova_sess_o}

\section{Referências}
\label{sec:nova_sess_o}




\onecolumn
\begin{usecase}
    \addtitle{Caso de Uso 1}{Exemplo de caso de uso}

    \addfield{Resumo:}{\lipsum[1]}

    \addfield{Ator Primario:}{Manolo}

    \addfield{Pré-condições:}{Aplicativo instalado}
\end{usecase}
\onecolumn

\onecolumn
\begin{figure}[h]
  \begin{center}
    \includegraphics[width=0.8\textwidth]{conteudo/lena}
    \caption{Clássica Lena}
  \end{center}
\end{figure}
\onecolumn

\onecolumn

\begin{longtable}{  c  L{1.5cm}  C{1.2cm}  R{2.5cm}  p{2cm}  m{3.5cm}  }
\caption{Entidade: Motorista}\\
\toprule
Nome & Tipo & Tamanho  & Restrições de Domínio & Regra de Derivação & Observações \\ \midrule
\rowcolor[gray]{0.9}
CNH & Integer & 11 & - & - & Campo que armazena o número da CNH \\
Nome & Char & 30 & - & - & Campo que armazena o nome do motorista \\
\rowcolor[gray]{0.9}
CPF & Char & 20 & Somente números & - & Campo que armazena o número do CPF \\
RG & Char & 20 & Somente números & - & Campo que armazena o número de RG \\ \bottomrule
\end{longtable}

\onecolumn

